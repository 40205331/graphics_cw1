%FILL THESE IN
\def\mytitle{Coursework Report}
\def\mykeywords{Fill, These, In, So, google, can, find, your, report}
\def\myauthor{Sam Reynolds}
\def\contact{40205331@live.napier.ac.uk}
\def\mymodule{Games Development (SET08116)}
%YOU DON'T NEED TO TOUCH ANYTHING BELOW
\documentclass[10pt, a4paper]{article}
\usepackage[a4paper,outer=1.5cm,inner=1.5cm,top=1.75cm,bottom=1.5cm]{geometry}
\twocolumn
\usepackage{graphicx}
\graphicspath{{./images/}}
%colour our links, remove weird boxes
\usepackage[colorlinks,linkcolor={black},citecolor={blue!80!black},urlcolor={blue!80!black}]{hyperref}
%Stop indentation on new paragraphs
\usepackage[parfill]{parskip}
%% all this is for Arial
\usepackage[english]{babel}
\usepackage[T1]{fontenc}
\usepackage{uarial}
\renewcommand{\familydefault}{\sfdefault}
%Napier logo top right
\usepackage{watermark}
%Lorem Ipusm dolor please don't leave any in you final repot ;)
\usepackage{lipsum}
\usepackage{xcolor}
\usepackage{listings}
%give us the Capital H that we all know and love
\usepackage{float}
%tone down the linespacing after section titles
\usepackage{titlesec}
%Cool maths printing
\usepackage{amsmath}
%PseudoCode
\usepackage{algorithm2e}

\titlespacing{\subsection}{0pt}{\parskip}{-3pt}
\titlespacing{\subsubsection}{0pt}{\parskip}{-\parskip}
\titlespacing{\paragraph}{0pt}{\parskip}{\parskip}
\newcommand{\figuremacro}[5]{
    \begin{figure}[#1]
        \centering
        \includegraphics[width=#5\columnwidth]{#2}
        \caption[#3]{\textbf{#3}#4}
        \label{fig:#2}
    \end{figure}
}

\lstset{
	escapeinside={/*@}{@*/}, language=C++,
	basicstyle=\fontsize{8.5}{12}\selectfont,
	numbers=left,numbersep=2pt,xleftmargin=2pt,frame=tb,
    columns=fullflexible,showstringspaces=false,tabsize=4,
    keepspaces=true,showtabs=false,showspaces=false,
    backgroundcolor=\color{white}, morekeywords={inline,public,
    class,private,protected,struct},captionpos=t,lineskip=-0.4em,
	aboveskip=10pt, extendedchars=true, breaklines=true,
	prebreak = \raisebox{0ex}[0ex][0ex]{\ensuremath{\hookleftarrow}},
	keywordstyle=\color[rgb]{0,0,1},
	commentstyle=\color[rgb]{0.133,0.545,0.133},
	stringstyle=\color[rgb]{0.627,0.126,0.941}
}

\thiswatermark{\centering \put(336.5,-38.0){\includegraphics[scale=0.8]{logo}} }
\title{\mytitle}
\author{\myauthor\hspace{1em}\\\contact\\Edinburgh Napier University\hspace{0.5em}-\hspace{0.5em}\mymodule}
\date{}
\hypersetup{pdfauthor=\myauthor,pdftitle=\mytitle,pdfkeywords=\mykeywords}
\sloppy
\begin{document}
	\maketitle
	\begin{abstract}
		This report will show the implementation of a graphics coursework. The coursework involves creating a 3D scene using OpenGL and showing the understanding of computer graphics principles. 
		
	\end{abstract}
    
	\textbf{Keywords -- Texturing, Material Shading, Lighting, Cameras, Transforms  }{}
    %START FROM HERE
	\section{Introduction}
    \paragraph{}
    The key effects being shown in the scene are texturing, applied to a skybox, plane and a dragon. Also the use of a point light to brighten up a specific area, in this case directly beneath the dragon.
    \paragraph{}
    Adding models to a scene giving them a terrain and a skybox looks visually interesting. Wrapping procedurally generated textures around models gives the dragon a realistic skin like texture and the skybox a custom world, in this case a hell based one, creates an incredibly realistic looking open world.
    
    \paragraph{}
    The difficulties within the project consist of a fair amount of complicated and frustrating maths. The maths involved in the implementation of shader files. As well as the maths, the algorithms implemented are quite technical and take time to fully grip.
    
    \figuremacro{h}{scene}{Hell Screenshot}{ - Screenshot of scene for part1}{1.0}
	
	\section{Related Work}
	\paragraph{}
	The resources used to build the scene mainly came from methods and techniques taught from the graphics workbook. Some extra features such as adding models and textures required searching a little on the internet but mainly came down to manipulating previously developed exercises.
	
	\paragraph{}
	The Idea from the scene came from an interest the dragon Alduin from the Skyrim game and the visually cool theme for the skybox fitted well.
	
	\figuremacro{h}{alduin}{Alduin}{ - Alduin in the Skyrim game}{1.0}
	 
	\section{Implementation}
	\paragraph{}
	The implementation of the scene came from techniques taught in the graphics workbook supported by the glm libraries creating the behind the scenes code for the project. Using the methods taught by the workbook exercises, the same process was followed to implement shaders, lighting, texturing etc.
	
	\paragraph{}
	Firstly a skybox was created and had a hell texture applied to it. This gave the scene an evil look with fiery coloured mountains and a sick coloured sky. Next a plane was introduced and wrapped with a default texture which will be updated in section 2. Finally a dragon model was rendered and added to the scene. The dragon had to be wrapped with its own matching skin texture that when finished looked of a very high quality.
	
	\figuremacro{h}{dragon_texture}{Texturing}{ - Skin texture applied to Alduin}{1.0}
	
	\paragraph{}
	The scene features a spot light directly underneath the dragon to light up the head of the dragon whilst leaving the rest of it mainly dark. The light can be turned on or off using keys. L will turn the light on and O will turn the light off.
	
	\paragraph{}
	The scene also features two cameras. The first of which is a free camera which can be used to move around and explore the whole scene using the mouse and keyboard. The mouse is used to look around whilst the WSAD keys are used to move forward, backwards, left and right. The second camera featured is a target camera. This camera gives the user a birds eye view of the plane which sits above the dragon. To switch from the free camera to the target camera, the k key is used.
	 
	\figuremacro{h}{target_camera}{Target Camera}{ - View from target camera}{1.0} 
    
	\section{Future Work}
    \paragraph{}
    Normal mapping is something which will be getting focussed on for the future of the scene, hopefully giving a river of fire effect or something similar. Shadowing would be the next thing to be introduced along with more lighting effects. Additional models can be expected alongside additional rendering techniques still yet to be taught.
    
    \paragraph{}
    As well as all of these extra effects the scene will be filled with extra objects and models. This time the models/objects will be transformed in a way that they will be either scaled, rotated, elevated and so on. When this is all in place a transform hierarchy will be added to the code to control all of these transforms.
	
	\section{Conclusion}	
	\paragraph{}
	This coursework was by far the hardest and most stressful of all coursework that has been assigned. However, the end results and final product is by far the coolest and most interesting out of all the other coursework assignments.
	
	\paragraph{}
	The scene achieved to have the realistic look and interesting model which was desired. The lighting effects, use of multiple cameras and texturing were also successfully applied within the project. With more time normal mapping and shadowing  would have been applied but unfortunately they were unsuccessfully featured in this scene. This will be used as a benchmark when it comes to further development.
	
	\paragraph{}
	The skills learned from this work will be very important in the work to come in the next section of the module and in future games modules.
		
	\end{document}
