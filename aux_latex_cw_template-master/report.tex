%FILL THESE IN
\def\mytitle{Coursework Report}
\def\mykeywords{Fill, These, In, So, google, can, find, your, report}
\def\myauthor{Sam Reynolds}
\def\contact{40205331@live.napier.ac.uk}
\def\mymodule{Games Development (SET08116)}
%YOU DON'T NEED TO TOUCH ANYTHING BELOW
\documentclass[10pt, a4paper]{article}
\usepackage[a4paper,outer=1.5cm,inner=1.5cm,top=1.75cm,bottom=1.5cm]{geometry}
\twocolumn
\usepackage{graphicx}
\graphicspath{{./images/}}
%colour our links, remove weird boxes
\usepackage[colorlinks,linkcolor={black},citecolor={blue!80!black},urlcolor={blue!80!black}]{hyperref}
%Stop indentation on new paragraphs
\usepackage[parfill]{parskip}
%% all this is for Arial
\usepackage[english]{babel}
\usepackage[T1]{fontenc}
\usepackage{uarial}
\renewcommand{\familydefault}{\sfdefault}
%Napier logo top right
\usepackage{watermark}
%Lorem Ipusm dolor please don't leave any in you final repot ;)
\usepackage{lipsum}
\usepackage{xcolor}
\usepackage{listings}
%give us the Capital H that we all know and love
\usepackage{float}
%tone down the linespacing after section titles
\usepackage{titlesec}
%Cool maths printing
\usepackage{amsmath}
%PseudoCode
\usepackage{algorithm2e}

\titlespacing{\subsection}{0pt}{\parskip}{-3pt}
\titlespacing{\subsubsection}{0pt}{\parskip}{-\parskip}
\titlespacing{\paragraph}{0pt}{\parskip}{\parskip}
\newcommand{\figuremacro}[5]{
    \begin{figure}[#1]
        \centering
        \includegraphics[width=#5\columnwidth]{#2}
        \caption[#3]{\textbf{#3}#4}
        \label{fig:#2}
    \end{figure}
}

\lstset{
	escapeinside={/*@}{@*/}, language=C++,
	basicstyle=\fontsize{8.5}{12}\selectfont,
	numbers=left,numbersep=2pt,xleftmargin=2pt,frame=tb,
    columns=fullflexible,showstringspaces=false,tabsize=4,
    keepspaces=true,showtabs=false,showspaces=false,
    backgroundcolor=\color{white}, morekeywords={inline,public,
    class,private,protected,struct},captionpos=t,lineskip=-0.4em,
	aboveskip=10pt, extendedchars=true, breaklines=true,
	prebreak = \raisebox{0ex}[0ex][0ex]{\ensuremath{\hookleftarrow}},
	keywordstyle=\color[rgb]{0,0,1},
	commentstyle=\color[rgb]{0.133,0.545,0.133},
	stringstyle=\color[rgb]{0.627,0.126,0.941}
}

\thiswatermark{\centering \put(336.5,-38.0){\includegraphics[scale=0.8]{logo}} }
\title{\mytitle}
\author{\myauthor\hspace{1em}\\\contact\\Edinburgh Napier University\hspace{0.5em}-\hspace{0.5em}\mymodule}
\date{}
\hypersetup{pdfauthor=\myauthor,pdftitle=\mytitle,pdfkeywords=\mykeywords}
\sloppy
\begin{document}
	\maketitle
	\begin{abstract}
		This report will show the implementation of a graphics coursework. The coursework involves creating a 3D scene using OpenGL and showing the understanding of computer graphics principles. 
		
	\end{abstract}
    
	\textbf{Keywords -- Texturing, Material Shading, Lighting, Cameras, Transforms  }{}
    %START FROM HERE
	\section{Introduction}
    \paragraph{}
    The key effects being shown in the scene are texturing, applied to a skybox, plane and a dragon. Also the use of a point light to brighten up a specific area, in this case directly beneath the dragon.
    \paragraph{}
    Adding models to a scene giving them a terrain and a skybox looks visually interesting. Wrapping procedurally generated textures around models gives the dragon a realistic skin like texture and the skybox a custom world, in this case a hell based one, creates an incredibly realistic looking open world.
    
    \paragraph{}
    The difficulties within the project consist of a fair amount of complicated and frustrating maths. The maths involved in the implementation of shader files. As well as the maths, the algorithms implemented are quite technical and take time to fully grip.
    
    \figuremacro{h}{scene}{Hell Screenshot}{ - Screenshot of scene for part1}{1.0}
	
	\section{Related Work}
	\paragraph{}
	The resources used to build the scene mainly came from methods and techniques taught from the graphics workbook. Some extra features such as adding models and textures required searching a little on the internet but mainly came down to manipulating previously developed exercises.
	
	\paragraph{}
	The Idea from the scene came from an interest the dragon Alduin from the Skyrim game and the visually cool theme for the skybox fitted well.
	
	\figuremacro{h}{alduin}{Alduin}{ - Alduin in the Skyrim game}{1.0}
	 
	\subsection{LineBreaks}
	Here is a line
    
    Here is a line followed by a double line break.
	This line is only one line break down from the above, Notice that latex can ignore this
    
    We can force a break \\ with the break operator.
    
	\subsection{Maths}
    Embedding Maths is Latex's bread and butter    
    
    {\centering \Large \(
        J = \begin{bmatrix}
            \frac{\delta e}{\delta \theta _0}
            \frac{\delta e}{\delta \theta _1}
            \frac{\delta e}{\delta \theta _2}
        \end{bmatrix}
        = e_{current} - e_{target} 
    \)\par}
	
	\subsection{Code Listing}
    You can load segments of code from a file, or embed them directly.
    
\begin{lstlisting}[caption = Hello World! in c++]
#include <iostream>

int main() {
    std::cout << "Hello World!" << std::endl;
    std::cin.get();
    return 0;
}
\end{lstlisting}

\lstinputlisting[caption = Hello World! in python script]{./sourceCode/hello.py}
    
\subsection{PseudoCode}

\begin{algorithm}[h]
\For{$i = 0$ \KwTo $100$}{
 print\_number = true\;
\If{i is divisible by 3}{
 print "Fizz"\;
 print\_number = false\;
}
\If{i is divisible by 5}{
 print "Buzz"\;
 print\_number = false\;
}
\If{print\_number}{
    print i\;
}
print a newline\;
}
\caption{FizzBuzz}
\end{algorithm}
	
\section{Conclusion}	
\bibliographystyle{ieeetr}
\bibliography{references}
		
\end{document}
